%%%%%%%%%%%%%%%%%%%%%%%%%%%%%%%%%%%%%%%%%%%%%%%%%%%%%%%%%%%%%%%
\documentclass[11pt]{article}

\usepackage[margin=1in]{geometry} 
\usepackage{amsmath,amsthm,amssymb}

\usepackage{listings}
\usepackage[dvipsnames]{xcolor}

\newenvironment{EX}[2][Exercise]{\begin{trivlist}
\item[{\color{red} \hskip \labelsep {\bfseries #1}\hskip \labelsep {\bfseries #2.}}]}{\end{trivlist}}

\newenvironment{SL}[1][Solution]{\begin{trivlist}
\item[{\color{blue} \hskip \labelsep {\bfseries #1:}}]}{\end{trivlist}}

\lstdefinestyle{mystyle}{
    backgroundcolor=\color{Gray!25!white},
    commentstyle=\color{PineGreen},
    keywordstyle=\color{Red},
    basicstyle=\footnotesize\ttfamily,
    breakatwhitespace=false,
    showstringspaces=false,
    showspaces=false,
    showtabs=false,
    breaklines=true,
    keepspaces=true,
    captionpos=b,
    numbers=left,
    numbersep=5pt,
    tabsize=2
}
\lstset{style=mystyle}

\begin{document}

% --------------------------------------------------------------
% Start here
% --------------------------------------------------------------

\noindent Matr\'icula (First-name Last-name) \hfill {\Large \bfseries CK0255: Homework A} \\

\begin{EX}{A.1}\
You are given a random sample $X_1,X_2,\dots,X_n$ from a population with PDF $f(x|\theta)$. Show that the maximising the likelihood function $\mathcal{L}(\theta|\mathbf{x})$, as a function of $\theta$ is equivalent to maximising $\log{[\mathcal{L}(\theta|\mathbf{x})]}$.
\end{EX}

\begin{SL}\
The $\log$ function is a strictly monotone increasing function. Therefore, $\mathcal{L}(\theta|\mathbf{x}) > \mathcal{L}(\theta'|\mathbf{x})$ if and only if $\log{[\mathcal{L}(\theta|\mathbf{x})]} > \log{[\mathcal{L}(\theta'|\mathbf{x})]}$.
\vskip0.250cm
\noindent
So, the value $\hat{\theta}$ that maximises $\log{[(\theta|\mathbf{x})]}$ is the same as the value that maximises $\mathcal{L}(\theta|\mathbf{x})$.
\end{SL}

% --------------------------------------------------------------
% Next exercise
% --------------------------------------------------------------

\begin{EX}{A.2}
Write code to generate $1000$ variables from the following distributions
\begin{enumerate}
\item $Y \sim \text{Binomial}(8,{2}/{3})$;
\item $Y \sim \text{Hyper-geometric}(\text{N}=10,\text{M}=8,\text{K}=4)$;\end{enumerate}
Compare the mean and the variance  with the theoretical values.
\end{EX}

\begin{SL}\
The $\texttt{R}$ code and corresponding output is the following
\begin{enumerate}
\item $\leadsto$
\begin{lstlisting}[language=S]
obs <- rbinom(1000,8,2/3) 
meanobs <- mean(obs) 
variance <- var(obs)
\end{lstlisting}

\begin{lstlisting}[]
Output:
 > meanobs
   [1] 5.231

 > variance
   [1] 1.707346
\end{lstlisting}
\item $\leadsto$
\begin{lstlisting}[language=S]
obs<- rhyper(1000,8,2,4) 
meanobs <- mean(obs) 
variance <- var(obs)
\end{lstlisting}

\begin{lstlisting}[language=S]
Output:
 > meanobs
   [1] 3.169

 > variance
   [1] 0.4488879
\end{lstlisting}
\end{enumerate}
\end{SL}

% --------------------------------------------------------------
% Stop here
% --------------------------------------------------------------

\end{document}