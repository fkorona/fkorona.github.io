%%%%%%%%%%%%%%%%%%%%%%%%%%%%%%%%%%%%%%%%%%%%%%%%%%%%%%%%%%%%%%%
\documentclass[11pt]{article}

\usepackage[margin=1in]{geometry} 
\usepackage{amsmath,amsthm,amssymb}

\usepackage{listings}
\usepackage[dvipsnames]{xcolor}

\newenvironment{EX}[2][Exercise]{\begin{trivlist}
\item[{\color{red} \hskip \labelsep {\bfseries #1}\hskip \labelsep {\bfseries #2.}}]}{\end{trivlist}}

\newenvironment{SL}[1][Solution]{\begin{trivlist}
\item[{\color{blue} \hskip \labelsep {\bfseries #1:}}]}{\end{trivlist}}

\lstdefinestyle{mystyle}{
    backgroundcolor=\color{Gray!25!white},
    commentstyle=\color{PineGreen},
    keywordstyle=\color{Red},
    basicstyle=\footnotesize\ttfamily,
    breakatwhitespace=false,
    showstringspaces=false,
    showspaces=false,
    showtabs=false,
    breaklines=true,
    keepspaces=true,
    captionpos=b,
    numbers=left,
    numbersep=5pt,
    tabsize=2
}
\lstset{style=mystyle}

\begin{document}

% --------------------------------------------------------------
% Start here
% --------------------------------------------------------------

\noindent Matr\'icula (First-name Last-name) \hfill {\Large \bfseries CK0030: Homework A} \\

\begin{EX}{A.1}
For a given function $f(x)$, the integral $\int_{a}^{b}{f(x)\text{d}x}$ computed using the formula
\begin{equation}
\int_{a}^{b}{f(x)\text{d}x} \approx h \Big[ \cfrac{1}{2}f(x_0) + \sum_{i=1}^{n-1}{f(x_i) + \cfrac{1}{2}f(x_n)} \Big], 
\end{equation}
is approximated by $n$ trapezoids of equal width $h$. 

\noindent Write a Python function that takes any $f$, $a$ and $b$, and $n$ as inputs and returns the approximation. 
\end{EX}

\begin{SL}\
We write a Python function \texttt{trapz.py} with variables corresponding to the notation
\begin{lstlisting}[language=Python]
def trapz(f, a, b, n):
    h = float(b-a)/n
    result = 0.5*f(a) + 0.5*f(b)  # 1st and 3rd term between brackets 
    for i in range(1, n):
        result += f(a + i*h)      # Loop through index i (2nd term)
    result *= h                   # Final multiplication
    return result
\end{lstlisting}
The function can be tested as follows
\begin{lstlisting}[language=Python]
>>> from trapz import trapz
>>> from math import exp
>>> v = lambda t: 3*(t**2)*exp(t**3)
>>> n = 4
>>> num_int = trapz(v, 0, 1, n)
>>> num_int
    1.9227167504675762
\end{lstlisting}
\end{SL}

% --------------------------------------------------------------
% Next exercise
% --------------------------------------------------------------

\begin{EX}{A.2} \

...
\end{EX}

\begin{SL} \

...
\end{SL}

% --------------------------------------------------------------
% Stop here
% --------------------------------------------------------------

\end{document}